\begin{tabular}{llrrrrrrrrr}
\toprule
 &  & \multicolumn{3}{r}{bias} & \multicolumn{3}{r}{rmse_att} & \multicolumn{3}{r}{std_att} \\
 & alpha & 0.333333 & 0.666667 & 1.000000 & 0.333333 & 0.666667 & 1.000000 & 0.333333 & 0.666667 & 1.000000 \\
T0 & N_co &  &  &  &  &  &  &  &  &  \\
\midrule
\multirow[t]{3}{*}{10} & 10 & 2.328 & 0.703 & 0.130 & 4.770 & 3.068 & 1.642 & 4.175 & 3.032 & 1.684 \\
 & 20 & 1.367 & 0.312 & 0.053 & 3.484 & 2.209 & 0.914 & 3.260 & 2.219 & 1.008 \\
 & 40 & 1.026 & 0.196 & 0.051 & 2.776 & 1.752 & 0.714 & 2.616 & 1.781 & 0.821 \\
\cline{1-11}
\multirow[t]{3}{*}{20} & 10 & 2.957 & 1.029 & 0.217 & 4.817 & 2.696 & 1.135 & 3.814 & 2.544 & 1.179 \\
 & 20 & 1.435 & 0.438 & 0.055 & 3.280 & 1.754 & 0.745 & 2.982 & 1.773 & 0.860 \\
 & 40 & 1.093 & 0.167 & 0.042 & 2.613 & 1.348 & 0.602 & 2.430 & 1.409 & 0.757 \\
\cline{1-11}
\multirow[t]{3}{*}{40} & 10 & 2.905 & 1.232 & 0.145 & 4.911 & 3.035 & 0.969 & 3.972 & 2.797 & 1.065 \\
 & 20 & 1.670 & 0.399 & 0.019 & 3.592 & 1.718 & 0.724 & 3.221 & 1.737 & 0.861 \\
 & 40 & 0.876 & 0.295 & 0.006 & 2.675 & 1.418 & 0.574 & 2.556 & 1.441 & 0.697 \\
\cline{1-11}
\bottomrule
\end{tabular}
